\documentclass{article}
\usepackage{listings}
\usepackage[colorlinks = true,
            linkcolor = blue,
            urlcolor  = blue,
            citecolor = blue,
            anchorcolor = blue]{hyperref}

\lstset{
    aboveskip=\bigskipamount,
    belowskip=\bigskipamount,
    basicstyle=\ttfamily,
}

\title{
    SNMPScan\\
    \large{Progetto Gestione di Reti anno 2016/17}
}
\author{Marco Cameriero}
\date{}

\providecommand{\inlinecode}[1]{\texttt{#1}}

% ------------------------------------------------------------------------------------------
% ------------------------------------------------------------------------------------------
% ------------------------------------------------------------------------------------------

\begin{document}



\maketitle



\inlinecode{SNMPScan} \`{e} un network scanner che ricerca agent SNMP.



\section{Introduzione}

SNMP \'{e} un protocollo relativamente semplice e leggero per la configurazione,
la gestione ed il monitoraggio degli apparati di rete. \'{E} un protocollo di livello
applicativo che utilizza la porta UDP/161.

\inlinecode{SNMPScan} scansiona una determinata subnet alla ricerca di agent SNMP gestibili.
L'applicazione effettua una scansione dell'intera subnet, inviando ad ogni singolo host una
\textit{Get} SNMP richiedendo il valore di \inlinecode{system.sysName.0}, che contiene il nome
dell'agent. Se l'agent risponde positivamente, viene considerato attivo, ed il suo nome
ed il suo indirizzo verranno stampati a schermo al termine dell'operazione.



\section{Implementazione e struttura interna}

\inlinecode{SNMPScan} utilizza la libreria \inlinecode{libsnmp}, parte del pacchetto
\inlinecode{net-snmp}, per la comunicazione con gli agent SNMP. La libreria si occupa
dell'implementazione del protocollo SNMP e della comunicazione con i singoli agent.

La logica principale dell'applicazione \'{e} all'interno della funzione \inlinecode{do\_scan},
la quale si aspetta un puntatore ad una struttura \inlinecode{struct scan\_state} ed il massimo
grado di parallelismo (quante richieste contemporanee effettuare al massimo). All'interno della
struttura \inlinecode{scan\_state} \'{e} contenuto tutto lo stato necessario per la scansione attuale,
tra cui la subnet da visitare, gli host trovati attivi e le richieste ancora pendenti. Le liste
sono semplici linked list implementate con delle macro in \inlinecode{list.h}.

Il ciclo principale effettua quante pi\'{u} richieste possibili (entro il limite impostato
dall'utente) e le memorizza in una coda. Dopodich\`{e} si sospende in attesa che almeno una di queste
abbia successo o vada in timeout o errore. Al ritorno dalla \inlinecode{select}, la funzione
\inlinecode{snmp\_read} o \inlinecode{snmp\_timeout} invoca il callback delle singole sessioni
per le quali \'{e} disponibile un aggiornamento di stato (errore o risposta). Notare come il codice
sia asincrono, ma comunque single thread. La funzione di callback \inlinecode{on\_snmp\_response},
a questo punto, aggiunge l'host alla lista di host attivi in caso di risposta positiva specificando
anche il valore dell'oid richiesto \inlinecode{system.sysName.0}, oppure lo ignora completamente
in caso di errore.

Terminata la scansione, stampa la lista di tutti gli host attivi con i relativi indirizzi e nomi.



\section{Test ed utilizzo}

\inlinecode{SNMPScan} supporta le seguenti opzioni:

\begin{itemize}
\item \inlinecode{-h}: Visualizza un aiuto veloce sulle opzioni.
\item \inlinecode{-p num}: Limita il massimo numero di richieste contemporanee.
      Il default \'{e} \inlinecode{4}.
\item \inlinecode{-d[d]}: Abilita i log per il debugging.
      Utilizzare \inlinecode{-dd} per un output pi\'{u} verboso.
\end{itemize}

Lo script \inlinecode{test.sh} pu\'{o} essere utilizzato per testare velocemente \inlinecode{SNMPScan}:
lo script avvia 6 container Docker con un agent SNMP ed esegue una scansione sulla subnet dei container.

\begin{lstlisting}
$ ./out/snmpscan 172.17.0.0/24

Scan results:
- Total hosts scanned: 255
- Hosts up: 6

Host 172.17.0.2:
    SNMPv2-MIB::sysName.0 = STRING: 0e60fcd56f61

Host 172.17.0.3:
    SNMPv2-MIB::sysName.0 = STRING: 3d110d32aaea

Host 172.17.0.4:
    SNMPv2-MIB::sysName.0 = STRING: 0d4eea195074

Host 172.17.0.5:
    SNMPv2-MIB::sysName.0 = STRING: 1ccfbdad74ab

Host 172.17.0.6:
    SNMPv2-MIB::sysName.0 = STRING: 4639c3c64453

Host 172.17.0.7:
    SNMPv2-MIB::sysName.0 = STRING: e3319d4903d4
\end{lstlisting}



\end {document}
